% Created 2017-08-24 Thu 09:01
% Intended LaTeX compiler: pdflatex
\documentclass[11pt]{article}
\usepackage[utf8]{inputenc}
\usepackage[T1]{fontenc}
\usepackage{graphicx}
\usepackage{grffile}
\usepackage{longtable}
\usepackage{wrapfig}
\usepackage{rotating}
\usepackage[normalem]{ulem}
\usepackage{amsmath}
\usepackage{textcomp}
\usepackage{amssymb}
\usepackage{capt-of}
\usepackage{hyperref}
\usepackage{minted}
\usepackage{color}
\usepackage{listings}
\usepackage[frenchb]{babel}
\author{bernard}
\date{}
\title{Serveur web dynamique en Java : les bases}
\hypersetup{
 pdfauthor={bernard},
 pdftitle={Serveur web dynamique en Java : les bases},
 pdfkeywords={},
 pdfsubject={},
 pdfcreator={Emacs 25.2.2 (Org mode 9.0)}, 
 pdflang={Frenchb}}
\begin{document}

\maketitle
\setcounter{tocdepth}{2}
\tableofcontents


\section{Pourquoi un serveur web en Java}
\label{sec:org6fe11ce}
\subsection{Pourquoi un serveur}
\label{sec:orge77076f}
Dans le cadre d'un système d'informations, on veut donner accès \emph{simultané} à des
informations à un nombre indéterminé d'utilisateurs.



\subsection{Pourquoi un serveur web}
\label{sec:org02b4d38}
Un serveur web est en fait :
\begin{itemize}
\item un serveur internet ou intranet, c'est-à-dire utilisant les protocoles de transport TCP/IP
\item utilisant le protocole de communication HTTP.
\end{itemize}

L'intérêt d'utiliser les protocoles TCP/IP est évidemment de tirer partie de
toute l'infrastructure réseau existante. L'intérêt d'utiliser HTTP est de tirer
parti de tous les logiciels/bibliothèques existantes, notamment les navigateurs web.

\subsection{Pourquoi un serveur web dynamique}
\label{sec:org7d2da29}

Les informations gérées par le S.I. (Système d'Informations) peuvent être
modifiées par les utilisateurs. En conséquence, il faut que le serveur puisse
générer dynamiquement (en cours d'utilisation) les informations à envoyer. À
contrario, un site web statique ne pourrait que servir des informations figées.

\subsection{Pourquoi en Java}
\label{sec:org72c1514}
Java est un langage populaire pour la réalisation de sites web dynamiques pour
plusieurs raisons :

\begin{description}
\item[{concurrence/parallélisme}] Il est possible de gérer les connections de
plusieurs utilisateurs en parallèle, tirant ainsi parti des architectures
multi-cœurs des ordinateurs.
\item[{performance}] En plus de tirer parti de tous les cœurs d'un ordinateur,
chacun de ceux-ci est utilisé efficacement grâce à la performance de la
JVM.
\item[{portabilité}] Grâce à la JVM, un serveur programmé en Java peut facilement
être déployé sur n'importe quelle architecture disposant d'une JVM. Cela
permet notamment d'externaliser l'hébergement du site (cf. \emph{cloud}, IaaS
voire PaaS).
\item[{fiabilité}] Le fait que Java soit un langage compilé à typage statique aide à
la réalisation de programmes fiables, même si les erreurs de compilation
\textbf{ne remplacent pas} les tests !
\item[{disponiblitié de bibliothèques/frameworks}] Le fait que les autres qualités
aient fait reconnaître l'intérêt d'utiliser Java pour la programmation de
serveurs web a provoqué le développement de nombreuses bibliothèques et
frameworks qui facilitent la programmation de serveurs.
\end{description}

\section{Principes de la programmations de serveurs dynamiques Web en Java}
\label{sec:orgb10e95a}

\subsection{TCP/IP et HTTP}
\label{sec:orgbd024f3}
TCP/IP permet de désigner une ressource à l'aide d'une URL. Cette URL indique :

\begin{itemize}
\item le protocole (HTTP/HTTPS)
\item le nom du serveur (qui permet d'obtenir l'adresse IP grâce au DNS)
\item le port (par défaut 80 pour un serveur web en HTTP et 443 pour le HTTPS)
\end{itemize}

Pour le port, on doit tenir compte du fait que les ports "de notoriété publique"
(\emph{well known ports}), et notamment ceux qui nous intéressent, ne peuvent être
utilisés par un simple utilisateur et requiert donc les privilèges
administrateur. En pratique, on développera en utilisant un autre port
disponible pour les simples utilisateurs et libre. On rendra paramétrable au
déploiement le port utilisé par notre serveur. En effet, il ne peut y avoir
qu'un seul programme en charge d'un port donné sur un machine donnée. Pour cette
raison, on devra s'assurer de terminer le programme en court d'exécution avant
de le relancer lors du cycle de développement.

\subsection{Servlet, containers et serveurs embarqués}
\label{sec:org443ec7b}

En Java, le composant de base de la programmation d'un serveur est une
\emph{Servlet}. Il y a deux façons de déployer :
\begin{itemize}
\item Sous la forme d'un WAR à déployer sur un container de servlets lui-même déjà déployé (par exemple un Tomcat)
\item Sous la forme d'un programme java autonome (JAR, le plus souvent "fat JAR"
incluant toutes les dépendances), qui contient lui-même le serveur (par
exemple Jetty).
\end{itemize}

De plus en plus, on préfère des livrables les plus autonomes possibles (par
exemple des machine virtuelles embarquant jusqu'au système d'exploitation !) et
par soucis de simplicité (pour ne pas avoir à installer de serveur Tomcat), on
commencera par utiliser un serveur Jetty embarqué. Ensuite, on utilisera un
framework qui permettra (notamment à l'aide de maven), de générer aussi bien un
fat JAR autonome qu'un WAR destiné à un container de Servlets.


\section{Embarquer un serveur Jetty}
\label{sec:org1146290}

Ci-dessous, les étapes nécessaires pour la réalisation d'un serveur web minimal consultable sur un port donné (i.e. 9092)

\subsection{Configuration}
\label{sec:org4830d3b}

On utilise Maven pour gérer les dépendances de nos projets. Pour réaliser un
serveur web embarquant Jetty, il suffit d'ajouter les dépendances \texttt{jetty-servlet} et \texttt{jetty-server} de \texttt{org.eclipse.jetty} dans
le fichier \texttt{pom.xml}. Par exemple :

\begin{minted}[frame=lines,fontsize=\scriptsize,xleftmargin=\parindent,linenos]{xml}
  <dependency>
    <groupId>org.eclipse.jetty</groupId>
    <artifactId>jetty-server</artifactId>
    <version>9.4.6.v20170531</version>
</dependency>
<dependency>
    <groupId>org.eclipse.jetty</groupId>
    <artifactId>jetty-servlet</artifactId>
    <version>9.4.6.v20170531</version>
</dependency>
\end{minted}

\subsection{Code de serveur Jetty élémentaire}
\label{sec:org1911ed8}

On peut créer un objet de classe \texttt{org.eclipse.jetty.server.Server} associé à un
port disponible quelconque (par exemple \texttt{9092}). Ensuite, on utilise un objet de
classe \texttt{org.eclipse.jetty.servlet.ServletHandler} pour gérer les servlets qui
implémenteront notre serveur. Ensuite, on peut associer, dans ce \texttt{handler}, une
classe implémentant l'interface
\href{https://docs.oracle.com/javaee/7/api/javax/servlet/Servlet.html}{Servlet} .
Ensuite, on lance le serveur dans un nouveau flux d'exécution concurrent et l'on
attend la fin de celui-ci :
\begin{minted}[frame=lines,fontsize=\scriptsize,xleftmargin=\parindent,linenos]{java}
package co.simplon.poleEmploi.server;


import org.eclipse.jetty.server.*;
import org.eclipse.jetty.servlet.*;

public class HelloServer {

   public static void main(String args[]) throws Exception{
          Server server = new Server(9092);
          ServletHandler handler = new ServletHandler();
          server.setHandler(handler);
          handler.addServletWithMapping(HelloGenericServlet.class, "/*");
          server.start();
          server.join();
   }
}
\end{minted}
\subsection{Code de Servlet HTTP élémentaire}
\label{sec:org87775d8}
Les servlets permettent d'implémenter tout types de serveurs, mais pour un
serveur HTTP, on utilisera plus précisément la classe
\href{https://docs.oracle.com/javaee/7/api/javax/servlet/GenericServlet.html}{javax.servlet.GenericServlet} . Il suffit alors de définir la méthode \href{https://docs.oracle.com/javaee/7/api/javax/servlet/GenericServlet.html#service-javax.servlet.ServletRequest-javax.servlet.ServletResponse-}{service}.
Les arguments sont la requête et la réponse à construire. Pour cette dernière,
on utilise le fait que l'argument passé est une référence : les modifications
effectuées sur l'objet de classe \href{https://docs.oracle.com/javaee/7/api/javax/servlet/ServletResponse.html}{javax.servlet.ServletResponse} seront
donc disponibles pour le code appelant.

\begin{minted}[frame=lines,fontsize=\scriptsize,xleftmargin=\parindent,linenos]{java}
package co.simplon.poleEmploi.server;

import java.io.IOException;

import javax.servlet.GenericServlet;
import javax.servlet.ServletException;
import javax.servlet.ServletRequest;
import javax.servlet.ServletResponse;

public class HelloGenericServlet extends GenericServlet {
  private static final long serialVersionUID = 1L;
  @Override
  public void service(ServletRequest request, ServletResponse response) throws ServletException, IOException {
        response.getWriter().println("Hello from HelloGenericServlet !\n"+"got:\n"+request);

  }

}
\end{minted}

Il suffit ensuite de se rendre à l'aide d'un navigateur sur l'adresse
\url{http://localhost:9092/} . On peut aussi utiliser n'importe quel autre client :
\begin{minted}[frame=lines,fontsize=\scriptsize,xleftmargin=\parindent,linenos]{sh}
wget --quiet http://localhost:9092/ -O /dev/stdout
\end{minted}
ou
\begin{minted}[frame=lines,fontsize=\scriptsize,xleftmargin=\parindent,linenos]{sh}
curl http://localhost:9092
\end{minted}

\subsection{Notion de cycle de vie (lifecycle)}
\label{sec:org5767318}

On remarque que l'argument de l'appel de la méthode \href{https://www.eclipse.org/jetty/javadoc/9.4.6.v20170531/org/eclipse/jetty/servlet/ServletHandler.html#addServletWithMapping-java.lang.Class-java.lang.String-}{addServletWithMapping} n'est
pas une instance de notre classe \texttt{HelloGenericServlet} mais une instance d'une
classe générique \href{http://docs.oracle.com/javase/8/docs/api/java/lang/Class.html?is-external=true}{Class} (!) qui représente une classe en Java, ici l'objet
représente notre classe \texttt{HelloGenericServlet}. Mais nous n'avons pas instancié
d'objet de cette classe avec un appel à \texttt{new}. C'est le serveur qui aura le
contrôle, \href{https://docs.oracle.com/javaee/7/tutorial/servlets002.htm}{entre autres}, de l'instanciation et de l'initialisation de l'objet
servlet.

\section{Serveur: inversion de contrôle, multithread}
\label{sec:org2048dfc}

\subsection{Inversion de contrôle}
\label{sec:org990e2c4}
Au niveau de l'architecture du code, il y a un changement fondamental entre les
programmes "classiques" en ligne de commande que nous avons implémentés jusqu'à
présent et les serveurs. En effet, lorsque les programmes "classiques" en ligne de
commande sont lancés, ils effectuent une tâche en demandant éventuellement à
l'environnement (utilisateur, disque, réseau, bases de données,…), des
informations avant de produire un résultat et de se terminer. Pour un serveur,
comme pour certaines applications avec une interface graphique, le programme n'a
pas l'initiative : il passe son temps à attendre des requêtes/évènements pour y
répondre. Le code qui implémente la réponse n'est pas appelé explicitement par
le programme principal, mais enregistré pour être appelé lorsqu'une requête/un
évènement survient. Parmi les conséquences de ce mécanisme, il y a le fait que les signatures
sont souvent prédéterminées. On pourrait être tenté de contourner cette
contrainte en utilisant des attributs en lecture et en écriture pour contourner
cette contrainte. Cependant, pour le cas des serveur, le fait que plusieurs
requêtes puissent arriver en même temps rend les modifications par \emph{effets de
bord} périlleuses, à cause des accès concurrent de la programmation multithread.


\subsection{Multithread, accès concurrents}
\label{sec:orgba098ff}

Pour des raisons de qualité de service, on voudra évidemment pouvoir répondre à
plusieurs requêtes en même temps. Cela pose un problème particulier si
l'implémentation modifie un état partagé, en raison d'accès concurrents. En
Java, on utilise le multithread pour implémenter de façon performante les
serveurs, mais il faut alors soit éviter de modifier un état global, soit
synchroniser ces modifications comme on va le constater.

\subsection{Implémentation d'une servlet avec effet de bord}
\label{sec:org84d311b}

On va implémenter une servlet qui modifie des compteurs \texttt{counterA} et
\texttt{counterB}, transférant le contenu d'un compteur à l'autre. À chaque requête, un
compteur est décrémenté et l'autre est incrémenté, donc leur somme doit rester
constante. Si ce n'est pas le cas, on est dans une situation "impossible" et
l'on compte le nombre d'appels où le serveur était dans une telle situation :

\begin{minted}[frame=lines,fontsize=\scriptsize,xleftmargin=\parindent,linenos]{java}
package co.simplon.poleEmploi.server;

import java.io.IOException;

import javax.servlet.GenericServlet;
import javax.servlet.ServletException;
import javax.servlet.ServletRequest;
import javax.servlet.ServletResponse;

public class StatefulServlet extends GenericServlet {
  private static final long serialVersionUID = 1L;
  public static final long SUM=10000;
  private long counterA= SUM;
  private long counterB=0;
  private boolean fromAToB= true;
  private long impossibleCounter=0;
  private long slowTransfert(final long v, final long delta) {
    System.err.println("slowTransfert called");
    return	v+delta;	
  }
    @Override
    public void service( ServletRequest request,
                          ServletResponse response ) throws ServletException,
                                                        IOException {
      if (fromAToB) {
          if(counterA > 0) {
            counterB=slowTransfert(counterB, 1);
            counterA=slowTransfert(counterA, -1);
          }else {
            fromAToB=false;
          }
        }else {
          if(counterB > 0) {
            counterA=slowTransfert(counterA, 1);
            counterB=slowTransfert(counterB, -1);
          }else {
            fromAToB= true;
          }
        }
      if((counterA+counterB) != SUM) {
          ++impossibleCounter;
        }
        response.getWriter().println("a= "+counterA+" b= "+counterB+" sum = "+
        (counterA+counterB)+"\n going from "+
        (fromAToB ? "A to B":"B to A")+"\n impossible count="+impossibleCounter+"\n");        
    }
}
\end{minted}

Si l'on se rend à l'adresse \url{http://localhost:9092/}, autant de fois que l'on
veut, le serveur semble fonctionner correctement. Mais si l'on décide de faire
un grand nombre de requêtes \emph{en parallèle}, les comptes ne sont plus justes et le
serveur est dans un état "impossible" :
\begin{minted}[frame=lines,fontsize=\scriptsize,xleftmargin=\parindent,linenos]{sh}
seq 100 |parallel -j 100 curl localhost:9092  &>/dev/null
\end{minted}

\begin{itemize}
\item Comment expliquer cela ?
\item Que se passe-t'il si l'on enlève l'appel à \mintinline[frame=lines,fontsize=\scriptsize,xleftmargin=\parindent,linenos]{java}{System.err.println("slowTransfert called");} ?
\item Que se passe-t'il si à la place, on execute une requête vers une base de données ?
\end{itemize}


\subsection{Résolutions de problèmes d'accès concurrents}
\label{sec:org4d57fa8}

Une première solution pourrait être de \emph{synchroniser} la méthode \texttt{service}. On
peut pour cela utiliser le mot clé \href{https://docs.oracle.com/javase/tutorial/essential/concurrency/syncmeth.html}{synchronized}, mais quelles seraient les
conséquences ?

Quelle solution doit-on utiliser pour avoir un serveur acceptable ?

\section{HTTP}
\label{sec:org944cbf0}

Lorsque l'on implémente un serveur HTTP, on peut utiliser (dériver de-) la
classe \href{https://tomcat.apache.org/tomcat-5.5-doc/servletapi/javax/servlet/http/HttpServlet.html}{javax.servlet.http.HttpServlet} plutôt que la classe
\href{https://tomcat.apache.org/tomcat-5.5-doc/servletapi/javax/servlet/GenericServlet.html}{javax.servlet.GenericServlet} afin d'avoir des méthodes plus spécifiques
correspondant aux différents \emph{verbes} ou \emph{méthodes} \href{https://fr.wikipedia.org/wiki/Hypertext_Transfer_Protocol#M.C3.A9thodes}{du protocole HTTP}, ainsi
qu'à des \href{https://tomcat.apache.org/tomcat-5.5-doc/servletapi/javax/servlet/http/HttpServletRequest.html}{requête} et \href{https://tomcat.apache.org/tomcat-5.5-doc/servletapi/javax/servlet/http/HttpServletResponse.html}{réponse} spécifique prenant en compte diverses \emph{méta-données}.


\subsection{Principales méthodes HTTP}
\label{sec:org40b1268}
On s'intéressera tout d'abord seulement aux méthodes HTTP utilisées pour
l'implémentation de sites/services web, en insistant sur celles qui sont
utilisables directement à partir de pages HTML: \texttt{GET} et \texttt{POST}.

\subsubsection{GET}
\label{sec:orgec39c94}

La méthode \texttt{GET} est celle qui est utilisée par un navigateur web lorsque l'on
visite une page web en indiquant une URL ou en cliquant sur un lien. Cela
correspond à la lecture des données associées à l'URL, sans modification des
informations stockées côté serveur.


\subsubsection{POST}
\label{sec:orga17ff40}

La méthode \texttt{POST} est celle qui est utilisée par un navigateur web lorsque l'on
envoie le contenu d'un formulaire. Cela correspond au cas général de l'envoi
d'informations qui vont avoir un effet côté serveur.

\subsubsection{PUT}
\label{sec:orgad00643}

La méthode \texttt{PUT} concerne elle l'envoi des données qui correspondent à l'URL, en
réciproque de \texttt{GET}. Cette méthode doit être \emph{idempotente}, c'est-à-dire qu'un
même \texttt{PUT} doit pouvoir être répété plusieurs fois sans que le résultat soit
différent côté serveur que si le \texttt{PUT} n'était fait qu'une seule fois. Pour
illustrer avec du code, l'instruction \mintinline[frame=lines,fontsize=\scriptsize,xleftmargin=\parindent,linenos]{java}{x= 4;} est
idempotente alors que \mintinline[frame=lines,fontsize=\scriptsize,xleftmargin=\parindent,linenos]{java}{x= x + 4;} ne l'est pas.

\subsubsection{DELETE}
\label{sec:orge22de11}

La méthode \texttt{DELETE} concerne elle la suppression des données qui sont à l'URL concernée.

\subsection{Principales méta-données HTTP}
\label{sec:org17f1f2b}

\subsubsection{Code de status}
\label{sec:org0848e2e}
La première information qu'on peut considérer comme méta-donnée est le code de
statut qui accompagne la réponse du serveur. Tout le monde connaît sans doute le
fameux code 404 qui indique qu'il n'y a rien à l'URL indiquée, et il y a \href{https://fr.wikipedia.org/wiki/Liste_des_codes_HTTP}{tout
une liste de codes pour différents cas de figure}. Lorsque l'on programme en
Java, on n'écrira bien sûr pas le code numérique mais l'on utilisera les
constantes nommées de la classe \href{https://tomcat.apache.org/tomcat-5.5-doc/servletapi/javax/servlet/http/HttpServletResponse.html}{javax.servlet.http.HttpServletResponse}, comme
\href{https://tomcat.apache.org/tomcat-5.5-doc/servletapi/javax/servlet/http/HttpServletResponse.html#SC_OK}{HttpServletResponse.SC\_OK} ou \href{https://tomcat.apache.org/tomcat-5.5-doc/servletapi/javax/servlet/http/HttpServletResponse.html#SC_INTERNAL_SERVER_ERROR}{HttpServletResponse.SC\_INTERNAL\_SERVER\_ERROR}.

\subsubsection{Entêtes (\emph{headers})}
\label{sec:orgb445f01}

Aussi bien la requête que la réponse peuvent contenir des informations sous la
forme de entêtes. La \href{https://en.wikipedia.org/wiki/List_of_HTTP_header_fields}{liste des entêtes} est longue d'autant qu'il est possible
d'ajouter des entêtes en plus de ceux qui sont standardisés. On s'intéressera
principalement aux suivants :

\begin{enumerate}
\item Content-Type
\label{sec:org5ab5cde}
Cet entête indique le type de contenu qui est renvoyé. Par exemple du HTML, qui
est une sous-catégorie du texte, et l'encodage, par exemple l'utf-8:
\texttt{Content-Type: text/html; charset=utf-8}.

\item Set-Cookie
\label{sec:org1b46d3b}
Cet entête permet d'enregistrer un \href{https://fr.wikipedia.org/wiki/Cookie_(informatique)}{cookie} sur le client. Pour ajouter un cookie
à une réponse, on n'utilisera pas de méthode générique de type \href{https://tomcat.apache.org/tomcat-5.5-doc/servletapi/javax/servlet/http/HttpServletResponse.html#addHeader(java.lang.String,\%2520java.lang.String)}{addHeader} sur la
réponse, mais la méthode spécifique \href{https://tomcat.apache.org/tomcat-5.5-doc/servletapi/javax/servlet/http/HttpServletResponse.html#addCookie(javax.servlet.http.Cookie)}{addCookie}. Il est possible d'ajouter
plusieurs cookies à une même réponse HTTP.
\end{enumerate}


\section{Contenu statique et contenu dynamique}
\label{sec:orgeab63ea}

Le plus souvent, le contenu à envoyer en réponse à une requête n'est pas
purement dynamique. Évidemment, on voudra mettre le moins possible de données
dans le code, donc on voudra pouvoir aussi renvoyer le contenu de fichiers : il
s'agit de contenu statique.

\subsection{DefaultServlet}
\label{sec:orga4a8617}

On peut envoyer le contenu de fichiers (ou de répertoires) correspondant à une
URL donnée en utilisant la class \href{https://www.eclipse.org/jetty/javadoc/current/org/eclipse/jetty/servlet/DefaultServlet.html}{DefaultServlet} fournie par Jetty. Il est
possible de permettre ou non de renvoyer le contenu des répertoires lorsque
l'URL correspond à un répertoire et non un fichier.

\subsection{WelcomeFiles}
\label{sec:orge41fb1f}
Plutôt que de renvoyer le contenu d'un répertoire lorsqu'aucun nom de fichier
n'est indiqué dans l'URL, on veut renvoyer le contenu d'un fichier. On parle
alors de \href{https://docs.oracle.com/cd/E19798-01/821-1841/bnaer/index.html}{Welcome Files}. On peut spécifier, par programmation (\href{https://www.eclipse.org/jetty/javadoc/9.4.6.v20170531/org/eclipse/jetty/server/handler/ContextHandler.html#setWelcomeFiles-java.lang.String:A-}{setWelcomeFiles})
ou par configuration (\href{https://www.javatpoint.com/welcome-file-list}{dans web.xml}), la liste ordonnée des noms de fichiers à
essayer pour trouver le contenu à renvoyer lorsqu'aucun fichier n'est indiqué
dans l'URL. On utilise pour cela le plus souvent le fichier \texttt{index.html} :
\begin{minted}[frame=lines,fontsize=\scriptsize,xleftmargin=\parindent,linenos]{java}
String [] welcomeFiles = {"index.html"};
context.setWelcomeFiles(welcomeFiles);
context.setResourceBase("./src/main/resources/");
\end{minted}

Comme on le voit, il ne faut pas oublier de spécifier le répertoire à partir
duquel rechercher les ressources, avec \href{https://www.eclipse.org/jetty/javadoc/9.4.6.v20170531/org/eclipse/jetty/server/handler/ContextHandler.html#setResourceBase-java.lang.String-}{setResourceBase} ou en utilisant
\href{https://www.eclipse.org/jetty/javadoc/9.4.6.v20170531/org/eclipse/jetty/servlet/Holder.html#setInitParameter-java.lang.String-java.lang.String-}{setInitParameter} sur le \texttt{ServletHolder} : \mintinline[frame=lines,fontsize=\scriptsize,xleftmargin=\parindent,linenos]{java}{setInitParameter("dirAllowed","true")} ou \mintinline[frame=lines,fontsize=\scriptsize,xleftmargin=\parindent,linenos]{java}{setInitParameter("dirAllowed","false")}.

\section{Utilisation d'une page HTML pour interagir avec le site}
\label{sec:orgdb90ebd}



\subsection{Liens et formulaires}
\label{sec:org81d70ba}
\subsubsection{Requête GET avec balise <a>}
\label{sec:orgff0afbd}

Il suffit d'utiliser une balise \texttt{a} avec un lien \texttt{href} vers l'URL pour laquelle
on a associé une servlet :
\begin{minted}[frame=lines,fontsize=\scriptsize,xleftmargin=\parindent,linenos]{html}
<a href="./dynamic">lien vers Servlet</a>
\end{minted}

\subsubsection{Requête POST avec un formulaire}
\label{sec:org96b359f}
On peut utiliser un formulaire avec le paramètre \texttt{method} à \texttt{"post"} et le
paramètre \texttt{action} indiquant l'URL pour laquelle on a associé une servlet :
\begin{minted}[frame=lines,fontsize=\scriptsize,xleftmargin=\parindent,linenos]{html}
<form action="./dynamic" method="post">
    <div>
        <label for="nom">Nom :</label>
        <input type="text" id="name" name="name" />
    </div>
<div>
    Password: <input type="password" name="password"/> <br/>
</div>
    <div>
        <label for="courriel">Courriel :</label>
        <input type="email" id="email" name="email"/>
    </div>
    <div>
        <label for="message">Message :</label>
        <textarea id="message" name="message"></textarea>
    </div>
    <div class="button">
        <button type="submit">Envoyer votre message</button>
    </div>
</form>
\end{minted}

Ensuite, on peut récupérer les paramètres côté serveur à partir de l'objet de
type \texttt{HttpServletRequest}, soit en récupérant le corps de la requête (\emph{body})
avec la méthode \href{https://tomcat.apache.org/tomcat-5.5-doc/servletapi/javax/servlet/ServletRequest.html#getReader()}{getReader()}, ou (exclusif !) en utilisant les
méthodes \href{https://tomcat.apache.org/tomcat-5.5-doc/servletapi/javax/servlet/ServletRequest.html#getParameterMap()}{getParameterMap()} ou \href{https://tomcat.apache.org/tomcat-5.5-doc/servletapi/javax/servlet/ServletRequest.html#getParameterNames()}{getParameterNames()} et \href{https://tomcat.apache.org/tomcat-5.5-doc/servletapi/javax/servlet/ServletRequest.html#getParameterValues(java.lang.String)}{getParameterValues(String
paramName)}.

\subsection{Javascript et méthodes HTTP}
\label{sec:orgee56324}
Si l'on veut utiliser d'autres méthodes (\texttt{PUT}, \texttt{DELETE}), il faut utiliser du code javascript.
Avec la bibliothèque \texttt{jQuery}, on peut par exemple écrire :
\begin{minted}[frame=lines,fontsize=\scriptsize,xleftmargin=\parindent,linenos]{html}
<script>
  $(document).ready(function(){
    $('#DeleteButton').click(function(){
      $.ajax({
          url: '/dynamic/data/123',
          method: 'DELETE'
        })
          .done(function( data ) {
            console.log(data);
          });
    });
  });
</script>

<input type="button" value="deleteValue" id="DeleteButton" >
\end{minted}

après avoir inclus la bibliothèque jQuery :
\begin{minted}[frame=lines,fontsize=\scriptsize,xleftmargin=\parindent,linenos]{html}
<script
        src="https://code.jquery.com/jquery-3.2.1.slim.min.js"
        integrity="sha256-k2WSCIexGzOj3Euiig+TlR8gA0EmPjuc79OEeY5L45g="
        crossorigin="anonymous"></script>
\end{minted}

Côté serveur, il faut bien sûr implémenter la méthode \href{https://tomcat.apache.org/tomcat-5.5-doc/servletapi/javax/servlet/http/HttpServlet.html#doDelete(javax.servlet.http.HttpServletRequest,\%2520javax.servlet.http.HttpServletResponse)}{doDelete}.

Pour voir le résultat, il faut afficher la console du navigateur.

On peut faire la même chose avec une méthode \texttt{PUT}.

\section{Documents dynamique}
\label{sec:org0f83d32}

On a vu qu'on pouvait retourner une chaîne de caractère construite dynamiquement
par une servlet, ou que l'on pouvait retourner le contenu d'un fichier. Parfois,
on voudra pouvoir générer dynamiquement des fragment d'un document dont la mise
en page sera, elle, statique. On utilise généralement pour cela des mécanismes
de \emph{templating}, par exemple avec de JSP (\emph{Java Server Pages}).

\section{Du site web dynamique à la webapp}
\label{sec:org33dac4f}

Si l'on désire implémenter un client \emph{CRUD} (\emph{Create}, \emph{Read}, \emph{Update}, \emph{Delete}):

\begin{itemize}
\item quelles méthodes HTTP semblent pertinentes ?
\item quelles URLs semblent pertinentes (cf RESTful) ?
\item quel format de données semble pertinent ?
\end{itemize}
\end{document}